\documentclass[a4paper]{article}

\usepackage[italian]{babel}

\usepackage{lipsum}

\usepackage{geometry}
\usepackage{graphicx}

\usepackage{titlesec} % per modificare lo stile dei titoli

\usepackage{enumitem} % Pacchetto che ci permette di indentare correttamente le liste numerate

\setlength{\parindent}{0pt} % Indentazione nulla per evitare di spostare i paragrafi a destra
\setlength\parskip{1em}     % Spaziatura tra paragrafi

% Pacchetti per il banner
\usepackage{graphicx}
\usepackage[table,xcdraw]{xcolor}
\usepackage{tikz}
\usepackage{tocloft}

\usepackage{float}

% Abilita numerazione e TOC fino al livello subparagraph

% Definizione hook per banner
\newcommand{\addbanner}{
  \AddToHook{shipout/background}{%
    \begin{tikzpicture}[remember picture, overlay]
      \node[anchor=north west, inner sep=0pt] at (current page.north west) {
        \includegraphics[height=\paperheight, width=7cm]{images/Vertical Banner.png}
      };
    \end{tikzpicture}
  }
}

\newcommand{\removebanner}{
  \RemoveFromHook{shipout/background}{%
    \begin{tikzpicture}[remember picture, overlay]
      \node[anchor=north west, inner sep=0pt] at (current page.north west) {
        \includegraphics[height=\paperheight, width=7cm]{images/Vertical Banner.png}
      };
    \end{tikzpicture}
  }
}

% Rende \paragraph simile a una "subsubsubsection"
\titleformat{\paragraph}[block]
  {\normalfont\normalsize\bfseries}{\theparagraph}{1em}{}

\titleformat{\subparagraph}[block]
  {\normalfont\normalsize\bfseries}{\thesubparagraph}{1em}{}

% Imposta spaziatura per paragrafi e subparagrafi
\titlespacing*{\paragraph}{0pt}{1.2ex plus .2ex}{1ex plus .2ex}
\titlespacing*{\subparagraph}{0pt}{1.2ex plus .2ex}{1ex plus .2ex}

% Impostiamo il titolo della tabella dei contenuti ad Indice
\renewcommand{\contentsname}{Indice}

\title{\textbf{Analisi dei requisiti della suite vivere}}
\author{} % Annulliamo l'autore e la data per fare spazio all'immagine
\date{}

\usepackage[hidelinks]{hyperref} % Serve per la table of contents cliccabile

\renewcommand\labelitemi{{\scriptsize\textbullet}} % Per le liste puntate con puntini più piccoli

% Definizione colori

\definecolor{darkblue}{HTML}{00468B}

\titleformat{\section}
  {\normalfont\color{darkblue}\bfseries\Large}
  {\thesection}{1em}{}

\renewcommand{\cftsecfont}{\color{darkblue}\bfseries}

\renewcommand{\cftsecpagefont}{\color{darkblue}\bfseries}

\newcommand{\addtitlebackground}{
  \AddToHookNext{shipout/background}{%
    \begin{tikzpicture}[remember picture, overlay]
      \fill[darkblue] (current page.south west) rectangle (current page.north east);
    \end{tikzpicture}
  }
}

\begin{document}

    \addtitlebackground

    \begin{titlepage}

    \color{white} % Tutto da qui in poi sarà bianco
    
    \newgeometry{left=2.5cm, right=2.5cm, top=2.5cm, bottom=2.5cm}

    \begin{figure}[t!]
        \centering
        \includegraphics[scale=0.10]{images/Unipa.png}
        \begin{center}
            \color{white}
            \par {UNIVERSITÀ DEGLI STUDI DI PALERMO}
            \par {DIPARTIMENTO DI INGEGNERIA}            
        \end{center}
    \end{figure}    

    \vspace*{.25cm} % padding sopra l'immagine
    \begin{center}
        \includegraphics[scale=.3]{images/Logo Dev Hub.png}\\[1cm]
        \vspace{.5cm} % Spazio sopra il titolo
        {\Huge \textbf{Analisi dei requisiti della suite vivere}}\\[1cm]
    \end{center}
        
    \vfill % Tutto ciò che segue va in fondo alla pagina

    \begin{minipage}[t]{0.45\textwidth}
        \textbf{Periti informatici} \\        
        Antonio Morabito \\
        Daniele Orazio Susino
    \end{minipage}%
    \hfill
    \begin{minipage}[t]{0.45\textwidth}
        \raggedleft                
        \textbf{Data ultimo aggiornamento} \\
        \today
    \end{minipage}

    \end{titlepage}

    \vspace{1cm}
    
    \newpage

    % TOC e Banner

    \newgeometry{left=7.5cm, right=2.5cm, top=2.5cm, bottom=2.5cm}
    \addbanner
    
    \tableofcontents
    
    % Torna al layout normale
    \removebanner
    \newgeometry{left=2.5cm, right=2.5cm, top=2.5cm, bottom=2.5cm}    

    \newpage

    \section{ Preambolo }

\subsection{ Introduzione }

Questo documento è stato redatto in modo tale da riuscire ad avere una \textbf{chiara visione del lavoro che verrà svolto}, in quanto con si è deciso di creare una \textbf{suite di applicativi} utili per la vita universitaria sia degli studenti che degli associati.

Ogni \textbf{progetto verrà descritto tramite un idea, i requisiti e lo stato} nella quale verge.

Questo \textbf{documento potrebbe essere aggiornato in futuro}, in quanto i requisiti dei software potrebbero cambiare inavvertitamente.

\subsection{ Requisiti speciali }

Esistono dei \textbf{requisiti} che sono \textbf{da applicare a tutti i servizi} della suite, verranno elencati di seguito.

Ogni servizio offerto dalla suite Vivere sarà \textbf{accessibile esclusivamente tramite un account registato}. Non sarà possibile accedere da guest. 

Per creare gli account \textbf{saranno accettate solamente mail UNIPA}, in quanto sono mail che associano univocamente la persona all'account e pertanto permettono di assicurarsi che la persona esista.

È assolutamente obbligatorio che i \textbf{servizi siano pensati in favore delle normative europee vigenti in ambito di protezione dei dati e della privacy delle persone (GDPR)}.
Questo implica il fatto che \textbf{non si possono memorizzare opinioni politiche delle persone} (anche se si tratta di elezioni universitarie).
Inoltre, i dati personali memorizzati dovranno essere minimali, ovvero essere memorizzati per uno scopo specifico, i dati che non sono strettamente utili a ciò che viene offerto dovranno essere inseriti in modo volontario e non obbligatorio (ergo non devono essere barriere per quanto riguarda l'accesso ai servizi offerti). 

Dovrà essere implementato per ogni servizio un \textbf{log-auditing in append e non modificabile, accessibile solo ai "Super Admin"}.

\color{red} I log saranno memorizzati per un tempo massimo di x Anni (?), da capire con Antonio questa cosa. \color{black}

\newpage

    \section { Risorse Umane (HR) }

\subsection{ Idea }

\par
{
    La piattaforma risorse umane è la piattaforma di gestione dei profili utente degli associati e degli studenti. \\
    Tutti gli studenti e gli associati dovranno registrarsi tramite questa piattaforma per poter usufruire della suite Vivere.
}

\subsection { Analisi dei requisiti }

\par{
    Gli utenti dovranno essere di 4 tipi:
    
    \begin{itemize}
        \item \textbf{Studente}: Utente che ha un account e quindi ha usufruito almeno una volta di un qualsiasi servizi della suite Vivere.
        \item \textbf{Staff}: Utente che è stato promosso da un amministratore a membro di staff (associato).
        \item \textbf{Amministratore}: Utente che è stato promosso da un amministratore ad amministratore.
        \item \textbf{Super Admin}: Utente che è già presente nelle migrazioni originali, ha gli stessi permessi dell'amministratore, ma i suoi permessi non possono essere revocati.
    \end{itemize}

    Deve essere previsto un amministratore per ogni corso, questo implica che una persona potrà essere amministratore anche di più corsi, siccome potrebbe capitare che prenda in gestione un corso simile dove ancora non ci sono associati.

    Dunque, una persona può far parte di più staff, questo per evitare che ci siano corsi senza staff e che quindi si blocchi l'accesso ai servizi vivere per quel determinato corso.

    Gli amministratori potranno modificare e bloccare gli utenti

    Nel caso una persona bannata prova a rifarsi l'account tramite la seconda mail istituzionale, il sistema deve bloccare in automatico la richiesta, riconoscendo giustamente che è la stessa persona (grazie al formato UNIPA).
    
    Gli amministratori potranno cambiare a piacimento il ruolo degli utenti.

    Durante la registrazione, bisognerà specificare:
    \begin{itemize}
        \item Nome e Cognome
        \item Anno di nascita (per usufruire di kaffettino)
        \item Corso
        \item Anno di corso: da prevedere anche FC, Laureando, Part Time, 4° superiore e 5° superiore, ma deve essere bloccato l'accesso ai minori di 16 anni.
        \item Numero di telefono
        \item Email: rigorosamente formato UNIPA, tranne se risulta come studente delle superiori
        \item Tag Telegram (opzionale)
        \item Tag Instagram (opzionale)
        \item Scuola di provenienza (opzionale, ma obbligatoria per gli staffer per la piattaforma orientamento)
        \item Password
    \end{itemize}

    Le password dovranno essere forti e non comuni, per evitare accessi indesiderati alla piattaforma, le specifiche:
    \begin{itemize}
        \item Minimo 12 caratteri
        \item Presente almeno un carattere speciale
        \item Presente almeno un numero
        \item Presente almeno una lettera maiuscola
        \item Presente almeno una lettera minuscola        
    \end{itemize}

    Alla login implementare rate limiting e invalidazione token, con hasing bcrypt.

    Effettuare il controllo per password comuni e possibilmente attaccabili tramite rainbow table.

    Ovviamente ogni utente può resettare password tramite una mail che gli arriva direttamente in casella, previa conferma dell'indirizzo email oscurato.

    Durante la registrazione di un utente verrà usata la mail come autenticazione a due fattori obbligatoriamente, in quanto serve a verificare che l'utente è veramente chi si dichiara di essere.
    
    Alla registrazione di un utente, il sistema deve effettuare uno scrape dei professori presenti nell'ateneo ed eventualmente mostrare accanto a quel nome un avviso se un match è stato trovato. Questo per evitare che i professori possano accedere ai servizi della piattaforma se non autorizzati da uno staffer.
    
    Gli utenti che si registreranno visualizzeranno una schermata di attesa nel frattempo che il loro account verrà accettato, siccome gli amministratori dovranno verificare l'iscrizione per renderla valida.

    Quando verrà effettuata una registrazione, verrà mandata una mail agli amministratori dello stesso corso dello studente che ha fatto la registrazione, esortandoli ad accettarlo.

    Un qualsiasi staffer può accettare una qualsiasi richiesta di registrazione.
    
    Il nome visualizzato e l'username per fare l'accesso sarà in formato unipa (es: MarioLuigi.Rossi03)

    A seguito di un ban, dovrà essere mandata una mail con la motivazione del ban (banalmente bisogna inserire un form dove si specifica la motivazione).

    In qualsiasi momento un utente può decidere di disiscriversi da eventuali mailing list oppure anche dalla piattaforma.

    Il super admin è un utente che non fa manutenzione regolarmente delle piattaforme, ma viene chiamato solamente per questioni di GDPR compliance, log auditing, ecc...

    Ogni utente non studente, può essere associato ad una o più aulette come gestore auletta. Egli riceverà diverse comunicazioni in merito alla gestione delle aulette a lui assegnate.

    Gli amministratori potranno aggiungere e rimuovere gestori auletta a piacimento. 
    
    Quando un utente verrà promosso a gestore auletta, gli dovrà arrivare una mail dove lo si esorta a venire nella sua auletta di riferimento a prendere le chiavi.

    Inoltre, quando verrà rimosso come gestore, verrà esortato a venire in auletta per restituire le chiavi.

    All'inizio dell'anno accademico gli studenti frequentanti le superiori verranno messi in una sezione "da confermare" dove bisognerà manualmente confermarli per farli passare di anno.

    Ogni inizio di anno accademico la piattaforma in automatico incrementerà l'anno accademico delle persone.

    In fase di registrazione, la piattaforma verificherà che l'età dello studente sia superiore o uguale a 16 anni. In caso contrario, la registrazione non sarà consentita.

    In fase di registrazione, si potrà scegliere se ricevere le email e per quali servizi. Ovviamente con opzione di cambiare preferenza in qualsiasi momento dal proprio profilo utente.

    Gli amministratori potranno specificare se un determinato utente possiede un ruolo istituzionale (ad esempio CDD). Ovviamente una persona può ricoprire più ruoli istituzionali.

    Magari si potrebbe fare in modo che il sito web prenda i ccs da questo DB?

    Ad ogni modo, per quanto riguarda i ruoli istituzionali, il sistema deve prevedere la data nella quale i rappresentanti decadono e di conseguenza archiviare il periodo di attività di rappresentanza se non rinnovato.

    Gli studenti, appena risulteranno agibili per avere una mail unipa, saranno obbligati a cambiarla.

    Il sistema dovrà verificare il corretto inserimento di nome e cognome facendo il confronto con la mail istituzionale.

    Gli utenti potranno ricevere svariate notifiche dai sistemi Vivere, consultabili dall'apposita sezione.

}

\subsection { Stato }

\par
{
    In implementazione
}

    \section { Problemi Tecnici }

\subsection{ Idea }

\par
{
    La piattaforma Problemi Tecnici deve \textbf{permettere agli utenti di segnalare un problema tecnico} ai manutentori delle applicazioni.
}

\subsection { Analisi dei requisiti }


\paragraph{Gli utenti}
\par
{
    Potranno segnalare dei problemi tecnici in un form, specificando:
    \begin{itemize}
        \item Titolo problema
        \item Descrizione problema
        \item Foto (anche multiple) o video del problema
        \item Applicazione del problema (Kaffettino, Eventi, ecc\ldots)
    \end{itemize}

    Chiaramente la piattaforma dovrà inviare anche l'email dell'utente che è stato a segnalare in modo che sia poi ricontattabile.
    Da li in poi si procederà per mail.
}

\paragraph{ Extra }
\par
{
    Sarebbe carino e più professionale, \textbf{trasformarla in una piattaforma di ticketing vera e propria}, magari integrando nel profilo utente tra le notifiche gli sviluppi.
    Sarebbe anche bello poter avere una chat asincrona direttamente in app e senza scambio di email se non per la notifica iniziale al gestore.
    Inoltre una sezione per vedere le problematiche segnalate ed il loro stato sarebbe interessante da prevedere. 
}

\subsection { Stato }

\par
{
    In ideazione
}

    \section { Kaffettino }

\subsection{ Idea }

\par
{
    Piattaforma che permette di organizzare l'inventario e la vendita di prodotti nelle aulette.
}

\subsection { Analisi dei requisiti }

\paragraph{Amministratori}

\par
{
    Gli amministratori potranno accedere alla dashboard dei resoconti e visualizzare i resoconti delle vendite e delle ricariche filtrando per giorno, settimana, mese, anno.

    I resoconti potranno essere filtrati come globali, per singole aulette, per utente.

    I resoconti di una determinata auletta dovranno essere accessibili solo da amministratori di un corso afferente a quell'auletta.

    Tutti i resoconti saranno scaricabili come Excel.

    Gli amministratori, tramite il portale, potranno ricaricare i conti degli utenti, ma non potranno ricaricare da soli il proprio conto, ci dovrà essere un secondo amministratore che lo carichi.

    Potranno anche assegnare e modificare la tessera di un utente.

    Potranno aggiungere prodotti nel magazzino (con l'opzione di aggiungere prodotti non presenti nel magazzino o crearne di nuovi).

    Una volta che i prodotti vengono creati non possono essere cancellati, siccome per storicità bisogna mantenere le compravendite degli utenti.

    Per rimuovere un prodotto dalla vendita, infatti, si potrà semplicemente delistare, ma continuerà ad esistere nella base di dati.

    Gli amministratori potranno anche modificare i nomi ed i costi dei prodotti, ma non si potrà cancellare.

    Potranno anche aggiungere, modificare e delistare aulette.

    Potranno anche modificare il debito massimo di ogni auletta (default 3 $\times$ valore del caffè).

    Qualora durante l'acquisto si supera il debito massimo, l'acquisto non va a buon fine segnalando il saldo insufficiente.

    All'inizio di ogni settimana, verrà inviata una mail dal sistema esortando di saldare il proprio debito a prescindere dalla quantità (anche se sei di un centesimo in debito in sostanza).

    Il debito vale per singolo magazzino, e non si potrà comprare nessun prodotto se pagando si sfora il debito massimo.

    Ogni prodotto dovrà avere i dati del fornitore: Nome, Indirizzo, N Telefono (opzionale), Email (opzionale).

    Gli amministratori potranno modificare la soglia di avviso per ogni prodotto, in modo tale che il sistema una volta raggiunta quella soglia mandi una mail ai gestori auletta.

    Gli amministratori potranno gestire dei coupon per dare una percentuale di sconto agli acquisti.

    I coupon non saranno cumulabili e saranno validi solamente per i prodotti scelti dagli amministratori, avranno inoltre una chiara data di scadenza.

    Gli utenti potranno abilitare lo sconto sul prossimo acquisto se dal portale lo abiliteranno.

}

\paragraph{ Utenti }

\par
{
    Ogni utente (studente, staff, admin) potrà effettuare la registrazione al servizio ma per avere il conto abilitato dovrà avere una carta associata.

    Qualora un utente prova a fare l'accesso al servizio ma non ha una carta associata, allora deve apparire il messaggio "Impossibile proseguire siccome non hai ancora una carta associata! Per favore vieni in auletta per avere la card".

    Gli utenti potranno richiedere una card sostitutiva in caso di smarrimento.

    Quando la card verrà assegnata riceveranno una mail che conferma che l'operazione è andata a buon fine.

    Ogni utente potrà visualizzare il proprio saldo.
    
    Ogni utente potrà visualizzare il proprio storico (data ed ora, importo, auletta di consumazione), spesa totale (oggi, settimana, mese, anno), spesa media (giornaliera, settimanale).

    Inoltre implementare il confronto con lo storico (questo mese +12\% rispetto alla media) e previsioni.
    
    Implementare un istogramma con tipologie di prodotti acquistati e quantità.

    Includere anche statistiche anonime ("bevi più caffè del 60\% degli studenti).

    Ovviamente per partecipare alle statistiche bisogna dare il consenso.

    Caffè gratis offerto al compleanno, con tanto di musichetta e messaggino.

    Il giorno del compleanno dovrà essere notificata la persona via mail facendogli gli auguri e spronandola a venire in auletta e riscattare il caffè gratuito.
    
    Il saldo dell'utente è spendibile solamente nel magazzino afferente all'auletta dove è stata ricaricata la card. Quindi per esempio, il saldo ricaricato ad ingegneria sarà spendibile ad ingegneria ed il saldo ad economia sarà spendibile solo ad economia.

    Il saldo fa riferimento al magazzino e non alle aulette perché così manteniamo le aulette del terzo piano e del deim come afferenti ad ingegneria.

    Gli amministratori potranno visualizzare le statistiche del proprio magazzino, con trend di consumo, previsioni di guadagni e di consumi ecc...

    Gli amministratori potranno anche impostare e recupere il pin della card dell'auletta nell'eventualità non vogliano che si usi per badgare .

}

\paragraph{ La piattaforma embedded }

\par
{
    L'embedded dovrà prevedere un modo per cambiare account del wifi per collegarsi ad unipa.
    
    Dovrà prevedere un modo per cambiare l'auletta di appartenenza (questo per fare in modo di poter spostare eventualmente l'embedded).
    
    Dovrà scaricare i prodotti del magazzino alla quale fa afferenza, e dovrà prevedere un modo per ricaricare la lista di prodotti senza accendere e spegnere.

    Dovrà riprodurre la canzone di buon compleanno al primo badge del compleanno di un utente, concedendogli un caffè gratis.

    In caso di acquisto, dovrebbe mostrare: "Grazie NomeUtente".

    Deve prevedere la mascotte Coffy che cambia espressione in base a quello che succede a schermo.

    Dovrà visualizzare i messaggi di errori specifici del backend (Nessuna connessione, prodotto non esistente, saldo insufficiente)

    In caso di errore critico, visualizzare una schermata di fallback con Coffy triste e con scritta di chiamare un amministratore (pensare a sensori non disponibili, errori critici di memoria).

}

\paragraph{ La piattaforma }
\par
{
    Nella pagina principale della piattaforma bisognerà allegare una foto diversa di "buongiornissimo kaffè!" diversa ad ogni caricamento.

    Le immagini devono essere, in ordine di priorità:
    \begin{itemize}
        \item Giorno del compleanno
        \item Debito
        \item Festività (Natale, Santa Lucia, Pasqua, Pasquetta, Capodanno, San Silvestro, Halloween, Morti, Epifania, Santo Stefano, Festa della donna, Festa della repubblica, Pride, Anniversario Ingegneria ed altre associazioni, Capodanno, Black Friday, Domenica delle palme, festa della liberazione, Festa della mamma, Festa del papà, Festa dei nonni, Compleanno di rosone, San francesco, San Valentino, Ferragosto, Giorno della memoria, ecc \ldots) 
        \item Giorno della settimana (Lunedì, martedì, mercoledì, \ldots)
    \end{itemize}

    La piattaforma dovrà segnalare agli amministratori quando le unità del magazzino scendono sotto la soglia, consigliando di fare rifornimento.
}

\paragraph{ Auditing }
\par
{
    Tutte le azioni devono essere loggate, in particolare le azioni amministrative:
    \begin{itemize}
        \item Ricariche, modifiche prezzi, assegnazioni card
    \end{itemize}

    I log non si potranno cancellare e verranno memorizzati direttamente in OS. Saranno di tipo append-only e non modificabili, al fine di garantire l’integrità storica delle operazioni.

    Bisogna distinguere anche gli sconti dovuti dal compleanno ecc.
}

\paragraph{ Extra }

\par
{
    L'embedded dovrà prevedere un modo per sbloccare un dispensatore di cialde.
    
    Dovrebbe prevedere delle animazioni durante il caricamento.

    È complicato ma sarebbe interessante prevedere un comportamento offline in modo da gestire gli acquisti in assenza di connessione, mettendo tutto in una coda di transazioni.
    Questo sarebbe utile per quei giorni dove la connessione UNIPA è congestionata, si potrebbe memorizzare tutto in memoria flash?

    Sarebbe ideale predisporre la piattaforma per accettare in futuro anche pagamenti tramite carta di credito per ricaricare il saldo.

}

\subsection { Stato }

\par
{
    In sviluppo, i componenti sono stati acquistati e si sta procedendo a creare la PCB ed il design finale.
}

    \section { Assistest }

\subsection{ Idea }

\par
{
    Piattaforma che permette alle aspiranti matricole di partecipare a simulazioni di prove TOLC e TOL.
    Look molto stile gamification
}

\subsection { Analisi dei requisiti }


\paragraph{Gli studenti}
\par
{
    \textbf{Importante} Come già detto in vivere HR, gli studenti non potranno registrarsi se non hanno almeno 16 anni.
    
    La matricola dovrà avere un'account valido (ruolo studente).

    Potranno partecipare a due tipi di simulazioni:
    
    \begin{itemize}
        \item In tempo reale con l'istruttore (Stile Kahoot)
        \item Nel tempo libero (stile quiz patente)
    \end{itemize}

    Potranno Partecipare alle simulazioni in tempo reale tramite codice alfanumerico di 6 caratteri a gruppi da 3.
    Potranno effettuare le simulazioni private scegliendo i panetti o i pacchetti di panetti alla quale sono interessati tutte le volte che vogliono, ricevendo alla fine il summary delle risposte date.

    Qualora si disconnettano durante una simulazione in tempo reale, allora verrà mantenuto lo score e si potranno riconnettere.
    Avranno la possibilità inoltre di scegliere un nickname per la singola partita.

    Potranno vedere i propri risultati per ogni simulazione fatta, con relative statistiche (media dei risultati, picco minimo e massimo) relative a tutte le simulazioni fatte e trend giornalieri, settimanali, mensili.
    Potranno visualizzare anche l'analisi delle aree di miglioramento (es. "devi migliorare in fisica")

    Potranno segnalare la risposta ad una domanda in modo tale che poi venga sistemata.
}

\paragraph{L'istruttore}
\par
{
    Qualsiasi membro di staff potrà fare l'istruttore.

    Potrà creare, modificare e cancellare panetti.
    Potrà creare, modificare e cancellare pacchetti di panetti.

    Un panetto è una raccolta di domande relative a un singolo argomento (ad esempio, 'Fisica', 'Matematica', ecc.), mentre un pacchetto di panetti consiste in più panetti raggruppati per argomenti o macrotemi (ad esempio, 'Fisica - Meccanica', 'Fisica - Termodinamica', ecc.).
    Ogni domanda dovrà prevedere anche una relativa risposta.

    Potrà creare, modificare e cancellare le simulazioni.
    Visualizzare un riepilogo dei risultati degli studenti per ogni simulazione.
    Visualizzare i risultati dei singoli studenti per ogni simulazione.
    Visualizzare i risultati dei singoli studenti per ogni simulazione priva degli studenti.

    Durante la creazione delle simulazioni, si dovranno scegliere i panetti di domande da inserire (o pacchetti di panetti).
    Ad esempio si potrà scegliere il panetto di panetti completo di una materia (ES. "Fisica"), oppure panetti su macro argomenti divisi per materia (Es. "Fluido dinamica", "Dinamica", "Termodinamica", "Cinematica", ecc\ldots)

    Qualora lo studente volesse contattare l'istruttore, la piattaforma dovrà prevedere un modo per mostrare il tag telegram/instagram o il contatto whatsapp dell'istruttore.

    Una volta finito l'assistest, lo studente verrà caricato in automatico sulla piattaforma HR sulla sezione matricole.

    Ogni panetto è composto da domande a risposta multipla, vero falso ed aperta. 
    Le domande aperte saranno usate esclusivamente nel caso di conti numerici da fare (attenzione a non essere case sensitive ed a specificare bene il formato), ovviamente con un certo grado di tolleranza ai calcoli effettuati.

    Gli istruttori potranno visualizzare sia i risultati generali di un'esercitazione, sia filtrandoli per utente, dando feedback direttamente tramite la piattaforma ad utenti che ne hanno bisogno, specificando la singola domanda.

    Gli istruttori potranno visualizzare le domande che sono state segnalate e potranno modificarle.

    Potranno dare il via alla simulazione stile kahoot manualmente, in modo tale da assicurarsi che tutti gli studenti siano entrati.

}

\paragraph{ La piattaforma }
\par
{
    Imporrà un limite di tempo per le simulazioni real time, ma non imporrà limiti per quelle private.

    Il codice alfanumerico per accedere alle lobby verrà generato sul momento, ovvero quando l'istruttore creerà la lobby.
}

\paragraph{ Extra }
\par
{
    Magari aggiungere delle streak stile duolingo che permettono di ottenere qualche bonus?
    Implementare sistema di punti in modo tale da poter fare una gara ed elargire premi al migliore?
    Si potrebbe implementare una leaderboard dove gli studenti possono vedere i progressi rispetto agli altri.
    La piattaforma sbloccherà contenuti multimediali in base ad una roadmap specificata dagli istruttori.
}

\subsection { Stato }

\par
{
    In ideazione
}

    \section { Drive }

\subsection{ Idea }

\par
{
    Utility che permette agli studenti di condividere appunti e materiale di studio.
}

\subsection { Analisi dei requisiti }

\par
{
    Tutti gli studenti potranno accedere alla piattaforma se sono stati verificati nel db.

    Gli studenti potranno visualizzare, scaricare e chiedere l'approvazione degli appunti.

    Questo vuol dire che i membri di staff dovranno controllare il materiale mandato dai ragazzi, potranno accettarlo o eliminarlo sempre con un commento costruttivo per fare capire agli studenti come contribuire alla comunità.

    I membri di staff potranno visualizzare, scaricare, aggiungere e cancellare gli appunti condivisi.

    Ogni file sarà identificato con:
    \begin{itemize}
        \item Nome del file
        \item Nome dell'autore (possibilità di lasciarlo anonimo)
        \item Data di caricamento
    \end{itemize}

    La piattaforma dovrà mantenere un log di elementi caricati, modificati e rimossi.

    La pagina iniziale dovrà permettere di selezionare qualsiasi macroarea (Ingegneria, Medicina, ecc...), dopo di che si potrà selezionare il corso, successivamente le materie (mettendo in una cartella a parte le materie non più erogate), subito dopo l'anno di erogazione ed infine la docenza.

    Ogni elemento dovrà avere dei tag che permetteranno di filtrare.
    
    I tag in questione indicheranno la tipologia: Appunti, Esercitazioni, Esercizi, Video (possibilmente anche link esterni tipo youtube), slide, documentazioni, etc \ldots

    Oltre ad un tag di tipologia ci saranno anche i tag di "ufficialità": Materiale del docente, materiale approvato dal docente, materiale non approvato dal docente.

    Inoltre ogni elemento avrà un ranking in termini di punteggio (stelle) dato dagli studenti, in questo modo gli studenti potranno sapere se in un determinato blocco di appunti manca qualcosa o ci sono degli errori, oppure se è ancora pertinente per poter preparare la materia.

    Insieme al punteggio, gli utenti potranno lasciare dei commenti e questi verranno moderati dagli amministratori, cancellandoli o approvandoli.

    Se dalla pubblicazione del commento passerà più di una settimana, il commento verrà approvato automaticamente.

    I file si potranno anche sostituire, in modo tale da aggiornare le criticità. 
    
    Per questo bisogna anche tenere traccia dei cambiamenti (stile git con i commenti dei commit).

    Gli utenti potranno anche flaggare dei commenti come offensivi, ed in quel caso potrebbero essere prese azioni disciplinari contro l'offendente.

}

\paragraph{ REGOLE }
\par
{
    Alcune regole per tutelare la piattaforma devono essere messe in atto, infatti chi trasgredirà questo regolamento potrebbe venire bannato da altri amministratori.
   
    \begin{itemize}
        \item Non sarà permesso caricare video-lezioni dei docenti e registrazioni senza consenso esplicito con email formale inoltrata ai gestori.
        \item Gli appunti non si potranno caricare come file word ma solo come PDF filigranato (vedesi utility di filigrana)
        \item Le dispense dei professori non si dovranno filigranare.
        \item Si dovrà promuovere un contesto di collaborazione ed ogni forma di hate-speech sarà punità con il ban.
    \end{itemize}
    
}

\paragraph{ EXTRA }
\par
{
    Sarebbe interessante implementare un sistema di moderazione AI dei commenti.
    Sarebbe interessante prevedere una barra di ricerca che permetta di navigare il file system più velocemente.
    Potrebbe essere carino incentivare il submitting di materiale magari dando accesso a sconti tramite kaffettino? O comunque dei premi / badge. In questo caso ogni studente potrebbe avere il proprio profilo e quindi li potremmo fare visualizzare i suoi contributi.
}

\subsection { Stato }

\par
{
    In ideazione
}

    \section { Eventi }

\subsection{ Idea }

\par
{
    La piattaforma eventi permetterà di comunicare la propria partecipazione a
    degli eventi specifici interni dell'associazione e di creare dei sondaggi.
}

\subsection { Analisi dei requisiti }

\par
{
    Ci saranno due tipi di sondaggi:
    \begin{itemize}
        \item Sondaggi "When To Meet": La piattaforma dovrà prevedere dei sondaggi per scegliere la data e l'orario di un evento. Pensare a https://timeful.app.
        \item Sondaggi "What About": Sondaggi dove si chiederanno una serie di domande, pensare a https://luma.com/?locale=it
    \end{itemize}

    I sondaggi potranno essere marchiati come "solo per staff" ed in quel caso potranno essere visualizzabili solo a membri di staff.
    
    Ogni evento può avere dei \textbf{documenti associati in allegato}.
}

\paragraph{ Gli amministratori }
\par
{
    Potranno creare modificare o annullare eventi.
    
    Potranno visualizzare le risposte ai sondaggi.
    
    Potranno decidere se in automatico aggiungere al calendario dell'intervistato l'evento tramite la piattaforma calendario.

    Potranno convocare ufficialmente alle assemblee ed ai vari eventi gli utenti, mandando comunicazione per notifica ed email. Sarebbe interessante anche per riunioni dei corsi ecc
}

\paragraph{ Gli utenti }
\par
{  
    Potranno rispondere ai sondaggi, vedere i sondaggi che hanno compilato e vedere le loro risposte.
    Inoltre potranno visualizzare i sondaggi alla quale possono rispondere.

    Gli utenti potranno inserire una nota opzionale quando confermano o meno la presenza ("Porto i panettoni", "Vengo 10 min in ritardo")

    Potranno vedere anche le statistiche su eventi alla quale si è partecipato (invitato in x eventi, partecipato in y eventi)
}

\paragraph{ Da vedere con di maio}
\par{
    Statistiche.
    Recensioni / Feedback.
    Collaboratori.
    Invitare ex partecipanti altri eventi
    Reminder mail.
    Poisizioni.
    Esportare in csv/excel.

    Confermare le entrate ed i pagamenti.
    Poter allegare file all'evento (resoconti) o mandare file a Tutti
    poter decidere quando e se re inviare notifiche a chi non ha dato conferma di partecipazione ad una cosa
    "tener segnato sia per l’associato che per chi gestisce chi deve pagare cose come magliette ed ecc"

}

\subsection { Stato }

\par
{
    In ideazione
}

    \section { Orientamento }

\subsection{ Idea }

\par
{
    Piattaforma che dovrà mantenere i contatti dei responsabili orientamenti delle scuole ed aiutare nell'organizzare gli orientamenti.
}

\subsection { Analisi dei requisiti }

\paragraph{ Gli amministratori }

\par
{
    Potranno aggiungere scuole al database.
    Potranno aggiungere i responsabili degli orientamenti delle scuole.
    
}

\paragraph{ Gli staffer }
\par
{
    Potranno farsi generare una mail da mandare ai responsabili delle scuole.
    
    Potranno creare, modificare o annulare gli orientamenti.

    Potranno dare disponibilità verso gli orientamenti.

    Gli potrà arrivare una mail in caso si crei un orientamento alla propria scuola di provenienza.


}

\paragraph{ Altro }
\par
{
    La piattaforma dovrà integrarsi con il calendario per mettere tutti gli orientamenti in calendario.
}

\subsection { Stato }

\par
{
    In ideazione
}

    \section { Orari }

\subsection{ Idea }

\par
{
    La piattaforma degli orari permetterà di scaricare l'orario di un corso.
}

\subsection { Analisi dei requisiti }

\paragraph{ Gli utenti }
\par
{
    Potranno utilizzare questa utility purché sia ovviamente registrato e verificato.

    Dovranno semplicemente selezionare un corso di studi e l'anno di frequenza, dopodiché il download del file immagine verrà avviato in automatico.

    Potranno anche selezionare delle materie a scelta di altri corsi di laurea.

    Potranno esportare l'orario in modo tale da poterlo inserire nel proprio calendario.

}

\paragraph{ Gli amministratori }
\par
{
    Potranno caricare dei layout e delle palette da utilizzare per ogni macroarea ("Ingegneria", "Medicina", "Economia", ecc...).
}

\paragraph{ La piattaforma }
\par
{
    Dovrà prendere il layout specificato dagli amministratori ed inserire al suo interno le informazioni del corso facendo uno scraping dal sito di UNIPA con gli orari effettivi.  
    L'orario dovrà prevedere anche le date di lezione, le date di sessione e le sospensioni.  
}

\subsection { Stato }

\par
{
    In ideazione
}

    \section { Aule Libere }

\subsection{ Idea }

\par
{
    La piattaforma delle aule libere servirà per visualizzare le aule libere dell'ateneo in una determinata data ed ora, oppure anche in un range di date ed ore.
}

\subsection { Analisi dei requisiti }

\paragraph{ La piattaforma }
\par
{
    Dovrà essere un wrapper per https://offertaformativa.unipa.it/offweb/public/aula/aulaCalendar.seam, ma offrendo qualche funzionalità in più.
    La piattaforma dovrà permettere di selezionare:
    \begin{itemize}
        \item Data (o range di date) nella quale si vuole prenotare l'aula
        \item Fascia oraria nella quale si vuole prenotare l'aula
        \item Quantità di posti necessari
        \item Tipologia di aula (Didattica, Didattica innovativa, Informatica, Impianto sportivo, Laboratorio, Sala Interna)
        \item Edificio
        \item Dipartimento
        \item Scuole (Formazione insegnanti, Scuola di medicina e chirurgia)
        \item Attributi dell'aula (Provvista di LIM, Provvista di lavagna, provvista di amplificazione, provvista di postazioni computer, in ristrutturazione)
        \item Se provvista di postazioni computer
    \end{itemize}

    Se verrà selezionata l'opzione "provvista di postazioni computer", allora si dovrà poter selezionare anche la quantità di postazioni necessaria

    Lo scaricamento iniziale delle aulee e dei loro attributi avverrà dal sito dell'università.

    La piattaforma dovrà poter lasciare modificare agli amministratori le aule.

    Questo vorrà dire:
    \begin{itemize}
        \item Creare, modificare e cancellare aulee esistenti
        \item Aggiungere, modificare e rimuovere gli attributi dell'aula
    \end{itemize}
    
    Una volta effettuata la ricerca, la piattaforma dovrà mostrare la lista di aule da poter selezionare.
    Le aule in ristrutturazione si dovranno visualizzare ma dovranno spuntare come non prenotabili.
    
    Una volta selezionata un aula, verrà visualizzata nella propria pagina:
    \begin{itemize}
        \item La posizione su una mappa leaflet.
        \item Gli orari nella quale l'aula è impegnata
    \end{itemize}
    
    Inoltre, la piattaforma, dovrà tenere traccia delle personalità dell'ateneo che si occupano di prenotare le aule nei vari dipartimenti.
    Gli amministratori, dovranno poter modificare queste personalità ed il loro dipartimenti di riferimento.
    Più personalità possono avere a che fare con un singolo dipartimento.
    La piattaforma quindi, dovrà preparare una mail da mandare alla personalità di riferimento afferente a quell'aula.
    La funzionalità della preparazione delle mail dovrà essere utilizzabile solo da membri di staff o amministratori.

}

\subsection { Stato }

\par
{
    In ideazione
}

    \section { QR }

\subsection{ Idea }

\par
{
    La piattaforma dei QR sarà esclusiva agli staffer e dovrà permettere di creare QR in diversi modi.
}

\subsection { Analisi dei requisiti }

\paragraph{ Gli staffer }
\par
{
    Potranno inserire un link o un testo generico che verrà trasformato in un QR code.
 
    Il QR potrà essere modificato in base a:
    \begin{itemize}
        \item Colore
        \item Forma dei punti
        \item Formato (SVG, PNG, JPEG)
        \item Dimensione (Se il formato non è SVG)
        \item Logo interno (Ci saranno tutti quelli delle sotto-associazioni)
    \end{itemize}
}

\subsection { Stato }

\par
{
    In ideazione
}

    \section { Calendario }

\subsection{ Idea }

\par
{
    La piattaforma calendario è pensata per un uso principalmente interno, a differenza della piattaforma eventi. 
    Permetterà di sapere quali sono gli eventi in associazione e di mostrarli in calendario.
}

\subsection { Analisi dei requisiti }

\paragraph{Gli amministratori}
\par
{
    Potranno aggiungere, modificare, cancellare gli inviti agli eventi.
    Un evento sarà caratterizzato da:
    \begin{itemize}
        \item Nome
        \item Giorno ed orario
        \item Tag ("CDD", "CCS", "Tutti", "Riunione di staff", ETC\ldots).
        \item Staff di appartenenza ("Ingegneria", "Economia", "Ateneo"\ldots)
        \item Priorità (?) probabilmente verrà abusata dagli amministratori
    \end{itemize}
}

\paragraph{Gli utenti}
\par
{
    Potranno registrarsi o cancellarsi dalle varie categorie di eventi.
    Quindi gli utenti potranno gestire gli eventi come meglio credono e potranno filtrare le tipologie di eventi che gli arrivano, ad esclusione di quelli dettati dalla proprià carica istituzionale.
}

\paragraph{ La Piattaforma }
\par
{
    Dovrà spedire gli inviti via mail ed essi dovranno essere compatibili con calendari google, outlook.

    
}

\paragraph{ La piattaforma embedded }
\par
{
    Sarà una semplice dashboard di visualizzazione calendario da mettere in auletta, si potrebbe tranquillamente fare con una vecchia raspberry pi ed uno schermo.
    Ovviamente questa sarebbe aggiornata in tempo reale con sync dalla piattaforma web
}

\paragraph{ EXTRA }
\par
{
    Sarebbe interessante cercare di capire se si possa automatizzare l'invio in base a ciò che già ci viene spedito (vedesi CDD).
}

\subsection { Stato }

\par
{
    In ideazione
}

    \section { Oggetti Smarriti }

\subsection{ Idea }

\par
{
    La piattaforma degli oggetti smarriti permetterà agli utenti di reclamare degli oggetti smarriti come propri.
}

\subsection { Analisi dei requisiti }

\paragraph{ La piattaforma }
\par
{
    
    La piattaforma si dovrà interfacciare con quella degli orari, in modo tale da dare un consiglio per quanto riguarda i corsi e la relativa annata nella quale fare girare l'annuncio.

    A proposito di annuncio, deve preparare un copy per quanto riguarda il mandare la comunicazione nei gruppi dei singoli anni.

    Dovrà arrivare una mail ai ragazzi se l'oggetto smarrito è possibile che sia stato perso esclusivamente nel loro corso di appartenenza, questo per evitare email spam.

}

\paragraph{ Gli amministratori }
\par
{
    Gli amministratori potranno inserire, modificare e cestinare un oggetto smarrito.

    Per ogni oggetto smarrito bisognerà specificare:
    \begin{itemize}
        \item Una foto dell'oggetto
        \item Una descrizione, idealmente con una caratteristica unica dell'oggetto
        \item Edificio ed aula in cui è stato trovato
        \item Un orario indicativo nella quale è stato perso.
        \item Dei tag (elettronica, documenti, materiale di cartoleria, ecc\ldots)
    \end{itemize}

    Potranno contrassegnare un oggetto come consegnato solo dopo che almeno una persona si è prenotata.

    In caso di cestinazione, non si cancellerà la entry nel db, ma verrà contrassegnata come cestinata (ovvereo che l'oggetto è stato gettato).

    Un amministratore può anche annullare un reclamo e rendere di nuovo l'oggetto reclamabile se la persona che ha effettuato il primo reclamo non si presenta oppure ha perso interesse nel reclamare l'oggetto.

    Ovviamente tutte le modifiche effettuate su un item dovranno essere loggate.
}

\par
{
    
    La piattaforma si dovrà interfacciare con quella degli orari, in modo tale da dare un consiglio per quanto riguarda i corsi e la relativa annata nella quale fare girare l'annuncio.

    A proposito di annuncio, deve preparare un copy per quanto riguarda il mandare la comunicazione nei gruppi dei singoli anni.

    Dovrà arrivare una mail ai ragazzi se l'oggetto smarrito è possibile che sia stato perso esclusivamente nel loro corso di appartenenza, questo per evitare email spam.

    Tutti gli oggetti saranno sempre listati nello storico anche se consegnati o cestinati, in modo tale da essere sempre rintracciabili.

    Dopo un mese da quando l'oggetto è stato trovato, la piattaforma manderà una mail agli amministratori dell'auletta afferente a quell'oggetto per spronarli a cestinare l'oggetto smarrito.

    Questo perché ovviamente se non verrà effettuato nessun reclamo l'oggetto andrà cestinato e quindi distrutto.

}


\paragraph{ Gli studenti }
\par
{
    Gli studedenti potranno visionare l'archivio degli oggetti smarriti in base all'auletta di afferenza del corso ed in base all'aula nella quale è stato ritrovato l'oggetto smarrito.

    Si potrà filtrare anche per parole chiave ("Astuccio", "Orologio", "Laptop").

    Potranno fare un reclamo e proclamare come proprio l'oggetto smarrito, verranno quindi esortati sia dalla piattaforma che da una mail automatica a presentarsi giorno tot per un ritiro.
}

\paragraph{ EXTRA }
\par
{
    Potrebbe essere interessante integrare il tutto con una mappa dell'università e le relative coordinate.
}

\subsection { Stato }

\par
{
    In ideazione
}

    \section { Elezioni }

\subsection{ Idea }

\par
{
    Piattaforma che permetterà ai membri di staff di calcolare, coefficientare e
    catalogare i voti.
}

\subsection { Analisi dei requisiti }

\par
{

    Si devono poter creare nuove elezioni e selezionare i corsi ai quali esse fanno riferimento, si possono inoltre selezionare interi dipartimenti o l'intera università.
    Deve essere possibile candidare un membro di staff ad una qualsiasi carica che però nel caso dei CCS possono essere solo relativi al proprio corso di laurea e dei CDD del proprio dipartimento.
    Deve essere possibile creare, cancellare e modificare delle nuove cariche (metti caso nasce ne nasce una nuova) e si deve poter decidere se possono partecipare studenti solo del proprio corso o del proprio dipartimento.
    Deve essere possibile creare, cancellare e modificare i poli universitari ed i corsi a loro afferenti.
    La piattaforma deve offrire una schermata di resoconto dei risultati adempiuti per ogni corso, dipartimento e polo universitario. (Chiaramente in base al fatto che le elezioni siano ).
    La visualizzazione deve essere suddivisa in corsi.
    Ad ogni votazione si dovranno scegliere le metodologie di votazione (D'Hondt, maggioritario, proporzionale, ecc), i candidati e le posizioni di questi ultimi. 
    Si deve prevedere un menù "candidati" dove poter inserire le candidature degli studenti, prevedendo tutte le cariche possibili (CCS, CDD, CSU, CNSU, CDA, CDA, ERSU) e le eventuali custom che si metteranno.
    I voti di staff dovranno essere categorizzati come tali.
    Sarà prevista anche una utility per organizzare giri aule.

    Sezione dove possiamo verificare se una lista di persone è inserita nel sistema.
}

\subsection { Stato }

\par
{
    In ideazione
}

    \section { Filigrana }

\subsection{ Idea }

\par
{
    Utility che permette di convertire file .docx in file .pdf e di filigranarli con tutti i loghi dell'associazione.
}

\subsection { Analisi dei requisiti }

\par
{
    La utility di filigrana è esclusiva ai membri di staff ed agli amministratori.
    
    Deve permettere di convertire file .docx in file .pdf e di filigranarli con tutti i loghi dell'associazione.
    
    Si potranno scegliere il logo (in base ai macrostaff) e la percentuale di opacità di quest'ultimo.

    Si potrà scegliere tra la versione della filigrana con il logo oppure solo con il testo.

    Si potranno caricare uno o più documenti contemporaneamente sia tramite selezione che come drag and drop.

    Qualora sia stato caricato più di un file, allora i file verranno zippati.

    L'opacità non dovrà mai essere inferiore del 5\%.

}

\subsection { Stato }

\par
{
    Implementazione completata, aspetto che mi vengano date le filigrane solo con il testo per pushare e mettere in up il sito definitivamente.
}

    \section { Segnalazioni }

\subsection{ Idea }

\par
{
    La piattaforma delle segnalazioni è una utility che permette di effettuare delle segnalazioni riguardanti le aulee, i dipartimenti, ed in generale gli spazi universitari.
}

\subsection { Analisi dei requisiti }

\paragraph{ Gli utenti }
\par
{
    Potranno effettuare delle segnalazioni che saranno identificate da:
    \begin{itemize}
        \item Foto o video breve della problematica
        \item Descrizione breve (Es. "LIM non funzionante").
        \item Descrizione lunga (Es. "La LIM in aula non funziona correttamente, spesso sfarfallando rende le lezioni impossibili").
        \item Aula o luogo (da capire se separare le due cose)
        \item Priorità: Bassa, Media, Alta, Pericolo
        \item Stato: Rifiutata, non segnalata, segnalata, sistemata
        \item Tipologia: Manutenzione ordinaria, manutenzione straordinaria, ecc $\ldots$
        \item Data segnalazione
    \end{itemize}

    Gli utenti potranno inserire solamente la foto o il video della segnalazione, le descrizioni e l'aula o il luogo.

    Queste segnalazioni dovranno poi essere revisionate e confermate da membri di staff.

    Ogni utente può vedere la lista di segnalazioni fatte nel proprio dipartimento, potendo inoltre filtrare per tutti i parametri della seganalzione detti prima.

    Chiaramente si dovrà poter vedere anche lo storico delle segnalazioni risolte
}

\paragraph{ Ogni staffer }
\par
{
    Ogni staffer potrà aggiungere, modificare e rifiutare una segnalazione.

    Durante la fase di accettazione di una segnalazione, lo staffer dovrà assegnare una priorità ed una tipologia alla segnalazione.

    Dovrà essere mantenuta un changelog per ogni segnalazione, ovviamente append-only.

    Per rifiutare una segnalazione, bisogna però inserire un commento obbligatorio in modo tale da giustificare il rifiuto.

    Potrà farsi generare una mail per poter segnalare le problematiche in base alle categorie, infatti, in genere è comodo raggruppare le segnalazioni e mandare una singola mail.
    
    Anche loro potranno vedere solo la lista di segnalazioni fatte nel proprio dipartimento, filtrando anche loro allo stesso modo degli utenti normali.
}

\paragraph{ Gli amministratori }
\par
{
    Potranno vedere tutte le segnalazioni di tutti i dipartimenti.

    Potranno filtrare in base ai dipartimenti

    Potranno aggiungere o modificare le personalità che si dovranno occupare di sistemare le segnalazioni ed a quale dipartimento fanno parte.
    Dovranno specificare di che tipologie di intervento si occupano le personalità.    
}

\paragraph{ La piattaforma }
\par
{
    La piattaforma, durante la fase di creazione di una segnalazione dovrà avvertire se esiste già una segnalazione simile.
    La piattaforma dovrà mantenere una lista di personalità che si occupano della manutenzione ed in generale di persone alla quale devono essere mandate le segnalazioni, in modo tale che in fase di segnalazione ufficiale il sistema indichi in base alla tipologia della segnalazione, la personalità alla quale mandare la mail.

    La piattaforma deve prevedere che in base alle tipologie di intervento ed all'edificio ci siano diverse personalità dell'ateneo alla quale fare riferimento per le segnalazioni (Signor Vincenzo, Fanale, \ldots).

    Come già detto, la piattaforma dovrà provvedere a generare/precompilare una mail per segnalare le problemmatiche, direttamente aprendo il client ed inserendo le email delle personalità interessate.

    Idealmente la piattaforma aggregherà quindi ogni problematica relativa ad una tipologia specifica in un unica mail, che poi il membro di staff potrà modificare e mandare.

    La piattaforma deve mandare delle mail ad ogni cambio di stato della segnalazione da loro fatta, sia per quanto riguarda la persona che ha sottomesso originariamente la segnalazione che per quella che ha mandato la mail.
    Quindi ad ogni cambio di stato (Rifiutata, Non segnalata, segnalata, sistemata) ci sarà una rispettiva mail
}

\paragraph{ Regole }
\par
{
    Alcune regole da rispettare saranno:
    \begin{itemize}
        \item Non si devono creare segnalazioni duplicate di altre
    \end{itemize}
}

\paragraph{ Extra }
\par
{
    Sarebbe carino implementare un sistema di analisi per quanto riguarda le statistiche di problematiche risolte, medie di problematiche del mese, aule più problematiche ecc. ecc.
}

\subsection { Stato }

\par
{
    In ideazione
}


    
\end{document}
