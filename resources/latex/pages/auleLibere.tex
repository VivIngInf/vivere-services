\section { Aule Libere }

\subsection{ Idea }

\par
{
    La piattaforma delle aule libere servirà per visualizzare le aule libere dell'ateneo in una determinata data ed ora, oppure anche in un range di date ed ore.
}

\subsection { Analisi dei requisiti }

\paragraph{ La piattaforma }
\par
{
    Dovrà essere un wrapper per https://offertaformativa.unipa.it/offweb/public/aula/aulaCalendar.seam, ma offrendo qualche funzionalità in più.
    La piattaforma dovrà permettere di selezionare:
    \begin{itemize}
        \item Data (o range di date) nella quale si vuole prenotare l'aula
        \item Fascia oraria nella quale si vuole prenotare l'aula
        \item Quantità di posti necessari
        \item Tipologia di aula (Didattica, Didattica innovativa, Informatica, Impianto sportivo, Laboratorio, Sala Interna)
        \item Edificio
        \item Dipartimento
        \item Scuole (Formazione insegnanti, Scuola di medicina e chirurgia)
        \item Attributi dell'aula (Provvista di LIM, Provvista di lavagna, provvista di amplificazione, provvista di postazioni computer, in ristrutturazione)
        \item Se provvista di postazioni computer
    \end{itemize}

    Se verrà selezionata l'opzione "provvista di postazioni computer", allora si dovrà poter selezionare anche la quantità di postazioni necessaria

    Lo scaricamento iniziale delle aulee e dei loro attributi avverrà dal sito dell'università.

    La piattaforma dovrà poter lasciare modificare agli amministratori le aule.

    Questo vorrà dire:
    \begin{itemize}
        \item Creare, modificare e cancellare aulee esistenti
        \item Aggiungere, modificare e rimuovere gli attributi dell'aula
    \end{itemize}
    
    Una volta effettuata la ricerca, la piattaforma dovrà mostrare la lista di aule da poter selezionare.
    Le aule in ristrutturazione si dovranno visualizzare ma dovranno spuntare come non prenotabili.
    
    Una volta selezionata un aula, verrà visualizzata nella propria pagina:
    \begin{itemize}
        \item La posizione su una mappa leaflet.
        \item Gli orari nella quale l'aula è impegnata
    \end{itemize}
    
    Inoltre, la piattaforma, dovrà tenere traccia delle personalità dell'ateneo che si occupano di prenotare le aule nei vari dipartimenti.
    Gli amministratori, dovranno poter modificare queste personalità ed il loro dipartimenti di riferimento.
    Più personalità possono avere a che fare con un singolo dipartimento.
    La piattaforma quindi, dovrà preparare una mail da mandare alla personalità di riferimento afferente a quell'aula.
    La funzionalità della preparazione delle mail dovrà essere utilizzabile solo da membri di staff o amministratori.

}

\subsection { Stato }

\par
{
    In ideazione
}