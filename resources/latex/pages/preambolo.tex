\section{ Preambolo }

\subsection{ Introduzione }

Questo documento è stato redatto in modo tale da riuscire ad avere una \textbf{chiara visione del lavoro che verrà svolto}, in quanto con si è deciso di creare una \textbf{suite di applicativi} utili per la vita universitaria sia degli studenti che degli associati.

Ogni \textbf{progetto verrà descritto tramite un idea, i requisiti e lo stato} nella quale verge.

Questo \textbf{documento potrebbe essere aggiornato in futuro}, in quanto i requisiti dei software potrebbero cambiare inavvertitamente.

\subsection{ Requisiti speciali }

Esistono dei \textbf{requisiti} che sono \textbf{da applicare a tutti i servizi} della suite, verranno elencati di seguito.

Ogni servizio offerto dalla suite Vivere sarà \textbf{accessibile esclusivamente tramite un account registato}. Non sarà possibile accedere da guest. 

Per creare gli account \textbf{saranno accettate solamente mail UNIPA}, in quanto sono mail che associano univocamente la persona all'account e pertanto permettono di assicurarsi che la persona esista.

È assolutamente obbligatorio che i \textbf{servizi siano pensati in favore delle normative europee vigenti in ambito di protezione dei dati e della privacy delle persone (GDPR)}.
Questo implica il fatto che \textbf{non si possono memorizzare opinioni politiche delle persone} (anche se si tratta di elezioni universitarie).
Inoltre, i dati personali memorizzati dovranno essere minimali, ovvero essere memorizzati per uno scopo specifico, i dati che non sono strettamente utili a ciò che viene offerto dovranno essere inseriti in modo volontario e non obbligatorio (ergo non devono essere barriere per quanto riguarda l'accesso ai servizi offerti). 

Dovrà essere implementato per ogni servizio un \textbf{log-auditing in append e non modificabile, accessibile solo ai "Super Admin"}.

\color{red} I log saranno memorizzati per un tempo massimo di x Anni (?), da capire con Antonio questa cosa. \color{black}

\newpage