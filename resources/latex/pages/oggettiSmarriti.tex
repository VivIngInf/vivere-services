\section { Oggetti Smarriti }

\subsection{ Idea }

\par
{
    La piattaforma degli oggetti smarriti permetterà agli utenti di reclamare degli oggetti smarriti come propri.
}

\subsection { Analisi dei requisiti }

\paragraph{ La piattaforma }
\par
{
    
    La piattaforma si dovrà interfacciare con quella degli orari, in modo tale da dare un consiglio per quanto riguarda i corsi e la relativa annata nella quale fare girare l'annuncio.

    A proposito di annuncio, deve preparare un copy per quanto riguarda il mandare la comunicazione nei gruppi dei singoli anni.

    Dovrà arrivare una mail ai ragazzi se l'oggetto smarrito è possibile che sia stato perso esclusivamente nel loro corso di appartenenza, questo per evitare email spam.

}

\paragraph{ Gli amministratori }
\par
{
    Gli amministratori potranno inserire, modificare e cestinare un oggetto smarrito.

    Per ogni oggetto smarrito bisognerà specificare:
    \begin{itemize}
        \item Una foto dell'oggetto
        \item Una descrizione, idealmente con una caratteristica unica dell'oggetto
        \item Edificio ed aula in cui è stato trovato
        \item Un orario indicativo nella quale è stato perso.
        \item Dei tag (elettronica, documenti, materiale di cartoleria, ecc\ldots)
    \end{itemize}

    Potranno contrassegnare un oggetto come consegnato solo dopo che almeno una persona si è prenotata.

    In caso di cestinazione, non si cancellerà la entry nel db, ma verrà contrassegnata come cestinata (ovvereo che l'oggetto è stato gettato).

    Un amministratore può anche annullare un reclamo e rendere di nuovo l'oggetto reclamabile se la persona che ha effettuato il primo reclamo non si presenta oppure ha perso interesse nel reclamare l'oggetto.

    Ovviamente tutte le modifiche effettuate su un item dovranno essere loggate.
}

\par
{
    
    La piattaforma si dovrà interfacciare con quella degli orari, in modo tale da dare un consiglio per quanto riguarda i corsi e la relativa annata nella quale fare girare l'annuncio.

    A proposito di annuncio, deve preparare un copy per quanto riguarda il mandare la comunicazione nei gruppi dei singoli anni.

    Dovrà arrivare una mail ai ragazzi se l'oggetto smarrito è possibile che sia stato perso esclusivamente nel loro corso di appartenenza, questo per evitare email spam.

    Tutti gli oggetti saranno sempre listati nello storico anche se consegnati o cestinati, in modo tale da essere sempre rintracciabili.

    Dopo un mese da quando l'oggetto è stato trovato, la piattaforma manderà una mail agli amministratori dell'auletta afferente a quell'oggetto per spronarli a cestinare l'oggetto smarrito.

    Questo perché ovviamente se non verrà effettuato nessun reclamo l'oggetto andrà cestinato e quindi distrutto.

}


\paragraph{ Gli studenti }
\par
{
    Gli studedenti potranno visionare l'archivio degli oggetti smarriti in base all'auletta di afferenza del corso ed in base all'aula nella quale è stato ritrovato l'oggetto smarrito.

    Si potrà filtrare anche per parole chiave ("Astuccio", "Orologio", "Laptop").

    Potranno fare un reclamo e proclamare come proprio l'oggetto smarrito, verranno quindi esortati sia dalla piattaforma che da una mail automatica a presentarsi giorno tot per un ritiro.
}

\paragraph{ EXTRA }
\par
{
    Potrebbe essere interessante integrare il tutto con una mappa dell'università e le relative coordinate.
}

\subsection { Stato }

\par
{
    In ideazione
}