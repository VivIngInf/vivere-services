\section { Eventi }

\subsection{ Idea }

\par
{
    La piattaforma eventi permetterà di comunicare la propria partecipazione a
    degli eventi specifici interni dell'associazione e di creare dei sondaggi.
}

\subsection { Analisi dei requisiti }

\par
{
    Ci saranno due tipi di sondaggi:
    \begin{itemize}
        \item Sondaggi "When To Meet": La piattaforma dovrà prevedere dei sondaggi per scegliere la data e l'orario di un evento. Pensare a https://timeful.app.
        \item Sondaggi "What About": Sondaggi dove si chiederanno una serie di domande, pensare a https://luma.com/?locale=it
    \end{itemize}

    I sondaggi potranno essere marchiati come "solo per staff" ed in quel caso potranno essere visualizzabili solo a membri di staff.
    
    Ogni evento può avere dei \textbf{documenti associati in allegato}.
}

\paragraph{ Gli amministratori }
\par
{
    Potranno creare modificare o annullare eventi.
    
    Potranno visualizzare le risposte ai sondaggi.
    
    Potranno decidere se in automatico aggiungere al calendario dell'intervistato l'evento tramite la piattaforma calendario.

    Potranno convocare ufficialmente alle assemblee ed ai vari eventi gli utenti, mandando comunicazione per notifica ed email. Sarebbe interessante anche per riunioni dei corsi ecc
}

\paragraph{ Gli utenti }
\par
{  
    Potranno rispondere ai sondaggi, vedere i sondaggi che hanno compilato e vedere le loro risposte.
    Inoltre potranno visualizzare i sondaggi alla quale possono rispondere.

    Gli utenti potranno inserire una nota opzionale quando confermano o meno la presenza ("Porto i panettoni", "Vengo 10 min in ritardo")

    Potranno vedere anche le statistiche su eventi alla quale si è partecipato (invitato in x eventi, partecipato in y eventi)
}

\paragraph{ Da vedere con di maio}
\par{
    Statistiche.
    Recensioni / Feedback.
    Collaboratori.
    Invitare ex partecipanti altri eventi
    Reminder mail.
    Poisizioni.
    Esportare in csv/excel.

    Confermare le entrate ed i pagamenti.
    Poter allegare file all'evento (resoconti) o mandare file a Tutti
    poter decidere quando e se re inviare notifiche a chi non ha dato conferma di partecipazione ad una cosa
    "tener segnato sia per l’associato che per chi gestisce chi deve pagare cose come magliette ed ecc"

}

\subsection { Stato }

\par
{
    In ideazione
}