\section { Calendario }

\subsection{ Idea }

\par
{
    La piattaforma calendario è pensata per un uso principalmente interno, a differenza della piattaforma eventi. 
    Permetterà di sapere quali sono gli eventi in associazione e di mostrarli in calendario.
}

\subsection { Analisi dei requisiti }

\paragraph{Gli amministratori}
\par
{
    Potranno aggiungere, modificare, cancellare gli inviti agli eventi.
    Un evento sarà caratterizzato da:
    \begin{itemize}
        \item Nome
        \item Giorno ed orario
        \item Tag ("CDD", "CCS", "Tutti", "Riunione di staff", ETC\ldots).
        \item Staff di appartenenza ("Ingegneria", "Economia", "Ateneo"\ldots)
        \item Priorità (?) probabilmente verrà abusata dagli amministratori
    \end{itemize}
}

\paragraph{Gli utenti}
\par
{
    Potranno registrarsi o cancellarsi dalle varie categorie di eventi.
    Quindi gli utenti potranno gestire gli eventi come meglio credono e potranno filtrare le tipologie di eventi che gli arrivano, ad esclusione di quelli dettati dalla proprià carica istituzionale.
}

\paragraph{ La Piattaforma }
\par
{
    Dovrà spedire gli inviti via mail ed essi dovranno essere compatibili con calendari google, outlook.

    
}

\paragraph{ La piattaforma embedded }
\par
{
    Sarà una semplice dashboard di visualizzazione calendario da mettere in auletta, si potrebbe tranquillamente fare con una vecchia raspberry pi ed uno schermo.
    Ovviamente questa sarebbe aggiornata in tempo reale con sync dalla piattaforma web
}

\paragraph{ EXTRA }
\par
{
    Sarebbe interessante cercare di capire se si possa automatizzare l'invio in base a ciò che già ci viene spedito (vedesi CDD).
}

\subsection { Stato }

\par
{
    In ideazione
}