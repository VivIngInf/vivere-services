\section { Risorse Umane (HR) }

\subsection{ Idea }

\par
{
    La piattaforma risorse umane è la \textbf{piattaforma di gestione dei profili utente degli associati e degli studenti}.
    \textbf{Tutti} gli studenti e gli associati \textbf{dovranno registrarsi tramite questa piattaforma} per poter usufruire della suite Vivere.
}

\subsection { Analisi dei requisiti }

\subsubsection{La piattaforma}

    La piattaforma dovrà permettere di registrarsi tramite form, specificando:
    \begin{itemize}
        \item \textbf{Nome e Cognome}
        \item \textbf{Anno di nascita} (dato necessario per kaffettino)
        \item \textbf{Corso} (dato necessario per il drive)
        \item \textbf{Anno di corso}: da prevedere anche FC, Laureando, Part Time, 4° superiore e 5° superiore, ma deve essere bloccato l'accesso ai minori di 16 anni. (Dato necessario per oggetti smarriti)
        \item \textbf{Numero di telefono}
        \item \textbf{Email}: rigorosamente formato UNIPA, tranne se risulta come studente delle superiori (Dato necessario per autenticazione)
        \item \textbf{Tag Telegram} (opzionale)
        \item \textbf{Tag Instagram }(opzionale)
        \item \textbf{Scuola di provenienza} (opzionale, ma obbligatoria per gli staffer per la piattaforma orientamento)
        \item \textbf{Password}
    \end{itemize}

    Le \textbf{password} dovranno essere forti e non comuni, per evitare accessi indesiderati alla piattaforma:
    \begin{itemize}
        \item Dovranno essere di \textbf{minimo 12 caratteri}
        \item Deve essere presente \textbf{almeno un carattere speciale}
        \item Deve essere presente \textbf{almeno un numero}
        \item Deve essere presente \textbf{almeno una lettera maiuscola}
        \item Deve essere presente \textbf{almeno una lettera minuscola}
    \end{itemize}

    La piattaforma e, per estensione tutte le altre, devono prevedere \textbf{4 tipologie di utente}: 
    \begin{itemize}
        \item \textbf{Studente}: Utente semplice senza particolari permessi.
        \item \textbf{Staff}: Utente che è stato promosso da un amministratore a membro di staff (associato), in determinate applicazioni avrà più poteri di uno studente normale, ma non sempre.
        \item \textbf{Amministratore}: Utente che si occupa delle configurazioni dei software della suite, spesso aggiungendo o rimuovendo dati organizzativi o interagendo con i profili degli utenti.
        \item \textbf{Super Admin}: Utente che è già presente nelle migrazioni originali, ha gli stessi permessi dell'amministratore, ma i suoi permessi non possono essere revocati. Saranno gli unici a poter acceddere ai log.
    \end{itemize}

    Verrà usata la \textbf{mail come autenticazione a due fattori}, obbligatoriamente. Questo perché serve a verificare che l'utente è veramente chi si dichiara di essere (grazie al formato UNIPA).

    Alla login implementare rate limiting e invalidazione token, con hasing bcrypt.

    Sempre alla login bisogna effettuare il controllo per password comuni e possibilmente attaccabili tramite rainbow table.

    Dopo la registrazione di un utente, \textbf{l'utente deve essere messo "in attesa"} tramite apposita schermata, visualizzabile anche quanddo l'utente prova a fare il login ddurante questa fase.
    
    Dopo che un utente si è registrato, \textbf{la piattaforma deve effettuare uno scrape dei professori presenti nell'ateneo ed eventualmente mostrare agli staffer, accanto a quel nome, un avviso se un match è stato trovato}. Questo per evitare che i professori possano accedere ai servizi della piattaforma se non autorizzati da uno staffer.
    
    Quando verrà effettuata una registrazione, verrà mandata una mail agli amministratori dello stesso corso dello studente che ha fatto la registrazione, esortandoli ad accettarlo.

    A seguito di un ban, la piattaforma dovrà mandare una mail con la motivazione del ban (banalmente bisogna inserire un form dove si specifica la motivazione).

    All'inizio dell'anno accademico gli studenti frequentanti le superiori verranno messi in una sezione "da confermare" dove bisognerà manualmente confermarli per farli passare di anno.

    Ogni inizio di anno accademico la piattaforma in automatico incrementerà l'anno accademico delle persone.

    In fase di registrazione, la piattaforma verificherà che l'età dello studente sia superiore o uguale a 16 anni. In caso contrario, la registrazione non sarà consentita.

    In fase di registrazione, si potrà scegliere se ricevere le email e per quali servizi. Ovviamente con opzione di cambiare preferenza in qualsiasi momento dal proprio profilo utente.

    Gli studenti, \textbf{appena risulteranno agibili per avere una mail unipa, saranno obbligati a cambiarla} (questo per evitare persone che usano all'infinito il proprio indirizzo mail).

    Il sistema dovrà verificare il corretto inserimento di nome e cognome facendo il confronto con la mail istituzionale.

\subsubsection{Gli utenti}

    Ogni utente può \textbf{resettare password} tramite una mail che gli arriva direttamente in casella, previa conferma dell'indirizzo email oscurato.

    Gli utenti che si registreranno visualizzeranno una \textbf{schermata di attesa nel frattempo che il loro account verrà accettato}, siccome gli amministratori dovranno verificare l'iscrizione per renderla valida.

    Il nome visualizzato e \textbf{l'username} per fare l'accesso sarà in \textbf{formato unipa} (es: MarioLuigi.Rossi03)

    In qualsiasi momento un utente può decidere di \textbf{disiscriversi da eventuali mailing list} oppure anche dalla piattaforma.

    Gli utenti potranno ricevere svariate notifiche dai sistemi Vivere, consultabili dall'apposita \textbf{sezione notifiche sul proprio profilo}.

    Quando un utente verrà promosso a gestore auletta, gli dovrà arrivare una mail dove lo si esorta a venire nella sua auletta di riferimento a prendere le chiavi.
    Inoltre, quando verrà rimosso come gestore, verrà esortato a venire in auletta per restituire le chiavi.

    Nel caso una \textbf{persona bannata} provi a \textbf{rifarsi l'account tramite la seconda mail istituzionale}, il sistema deve \textbf{bloccare in automatico la richiesta}, riconoscendo giustamente che è la stessa persona (grazie al formato UNIPA).

\subsubsection{Gli staffer}

    Un qualsiasi staffer può \textbf{accettare una qualsiasi richiesta di registrazione}.

    Ogni utente non studente, \textbf{può essere associato ad una o più aulette come gestore auletta}. Egli riceverà diverse comunicazioni in merito alla gestione delle aulette a lui assegnate.


\subsubsection{Gli amministratori}

    Gli amministratori potranno \textbf{cambiare a piacimento il ruolo degli utenti}.

    Deve essere previsto \textbf{almeno un amministratore per ogni corso}, questo implica che una persona potrà essere \textbf{amministratore anche di più corsi}, siccome potrebbe capitare che prenda in gestione un corso simile dove ancora non ci sono associati.

    Dunque, \textbf{una persona può far parte di più staff}, questo per evitare che ci siano corsi senza staff e che quindi si blocchi l'accesso ai servizi vivere per quel determinato corso.

    Gli amministratori potranno modificare e, in caso di comportamenti scorretti, bloccare gli utenti.

    Gli amministratori potranno \textbf{aggiungere e rimuovere gestori auletta a piacimento}. 
        
    Gli amministratori potranno \textbf{specificare se un determinato utente possiede un ruolo istituzionale} (ad esempio CDD). Ovviamente una persona può ricoprire \textbf{più ruoli istituzionali}.

\subsubsection{I super admin}
    Il super admin è un utente che non fa manutenzione regolarmente delle piattaforme, ma viene \textbf{chiamato solamente per questioni di GDPR compliance, log auditing, ecc}...

\subsubsection{Extra}

    Magari si potrebbe fare in modo che il \textbf{sito web prenda i ccs da questo DB}?
    Ad ogni modo, per quanto riguarda i ruoli istituzionali, il sistema deve \textbf{prevedere la data nella quale i rappresentanti decadono e di conseguenza archiviare il periodo di attività di rappresentanza se non rinnovato}.

\subsection { Stato }

    In implementazione
