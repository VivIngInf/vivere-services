\section { Assistest }

\subsection{ Idea }

\par
{
    Piattaforma che permette alle aspiranti matricole di partecipare a simulazioni di prove TOLC e TOL.
    Look molto stile gamification
}

\subsection { Analisi dei requisiti }


\paragraph{Gli studenti}
\par
{
    \textbf{Importante} Come già detto in vivere HR, gli studenti non potranno registrarsi se non hanno almeno 16 anni.
    
    La matricola dovrà avere un'account valido (ruolo studente).

    Potranno partecipare a due tipi di simulazioni:
    
    \begin{itemize}
        \item In tempo reale con l'istruttore (Stile Kahoot)
        \item Nel tempo libero (stile quiz patente)
    \end{itemize}

    Potranno Partecipare alle simulazioni in tempo reale tramite codice alfanumerico di 6 caratteri a gruppi da 3.
    Potranno effettuare le simulazioni private scegliendo i panetti o i pacchetti di panetti alla quale sono interessati tutte le volte che vogliono, ricevendo alla fine il summary delle risposte date.

    Qualora si disconnettano durante una simulazione in tempo reale, allora verrà mantenuto lo score e si potranno riconnettere.
    Avranno la possibilità inoltre di scegliere un nickname per la singola partita.

    Potranno vedere i propri risultati per ogni simulazione fatta, con relative statistiche (media dei risultati, picco minimo e massimo) relative a tutte le simulazioni fatte e trend giornalieri, settimanali, mensili.
    Potranno visualizzare anche l'analisi delle aree di miglioramento (es. "devi migliorare in fisica")

    Potranno segnalare la risposta ad una domanda in modo tale che poi venga sistemata.
}

\paragraph{L'istruttore}
\par
{
    Qualsiasi membro di staff potrà fare l'istruttore.

    Potrà creare, modificare e cancellare panetti.
    Potrà creare, modificare e cancellare pacchetti di panetti.

    Un panetto è una raccolta di domande relative a un singolo argomento (ad esempio, 'Fisica', 'Matematica', ecc.), mentre un pacchetto di panetti consiste in più panetti raggruppati per argomenti o macrotemi (ad esempio, 'Fisica - Meccanica', 'Fisica - Termodinamica', ecc.).
    Ogni domanda dovrà prevedere anche una relativa risposta.

    Potrà creare, modificare e cancellare le simulazioni.
    Visualizzare un riepilogo dei risultati degli studenti per ogni simulazione.
    Visualizzare i risultati dei singoli studenti per ogni simulazione.
    Visualizzare i risultati dei singoli studenti per ogni simulazione priva degli studenti.

    Durante la creazione delle simulazioni, si dovranno scegliere i panetti di domande da inserire (o pacchetti di panetti).
    Ad esempio si potrà scegliere il panetto di panetti completo di una materia (ES. "Fisica"), oppure panetti su macro argomenti divisi per materia (Es. "Fluido dinamica", "Dinamica", "Termodinamica", "Cinematica", ecc\ldots)

    Qualora lo studente volesse contattare l'istruttore, la piattaforma dovrà prevedere un modo per mostrare il tag telegram/instagram o il contatto whatsapp dell'istruttore.

    Una volta finito l'assistest, lo studente verrà caricato in automatico sulla piattaforma HR sulla sezione matricole.

    Ogni panetto è composto da domande a risposta multipla, vero falso ed aperta. 
    Le domande aperte saranno usate esclusivamente nel caso di conti numerici da fare (attenzione a non essere case sensitive ed a specificare bene il formato), ovviamente con un certo grado di tolleranza ai calcoli effettuati.

    Gli istruttori potranno visualizzare sia i risultati generali di un'esercitazione, sia filtrandoli per utente, dando feedback direttamente tramite la piattaforma ad utenti che ne hanno bisogno, specificando la singola domanda.

    Gli istruttori potranno visualizzare le domande che sono state segnalate e potranno modificarle.

    Potranno dare il via alla simulazione stile kahoot manualmente, in modo tale da assicurarsi che tutti gli studenti siano entrati.

}

\paragraph{ La piattaforma }
\par
{
    Imporrà un limite di tempo per le simulazioni real time, ma non imporrà limiti per quelle private.

    Il codice alfanumerico per accedere alle lobby verrà generato sul momento, ovvero quando l'istruttore creerà la lobby.
}

\paragraph{ Extra }
\par
{
    Magari aggiungere delle streak stile duolingo che permettono di ottenere qualche bonus?
    Implementare sistema di punti in modo tale da poter fare una gara ed elargire premi al migliore?
    Si potrebbe implementare una leaderboard dove gli studenti possono vedere i progressi rispetto agli altri.
    La piattaforma sbloccherà contenuti multimediali in base ad una roadmap specificata dagli istruttori.
}

\subsection { Stato }

\par
{
    In ideazione
}