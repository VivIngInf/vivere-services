\section { Elezioni }

\subsection{ Idea }

\par
{
    Piattaforma che permetterà ai membri di staff di calcolare, coefficientare e
    catalogare i voti.
}

\subsection { Analisi dei requisiti }

\par
{

    Si devono poter creare nuove elezioni e selezionare i corsi ai quali esse fanno riferimento, si possono inoltre selezionare interi dipartimenti o l'intera università.
    Deve essere possibile candidare un membro di staff ad una qualsiasi carica che però nel caso dei CCS possono essere solo relativi al proprio corso di laurea e dei CDD del proprio dipartimento.
    Deve essere possibile creare, cancellare e modificare delle nuove cariche (metti caso nasce ne nasce una nuova) e si deve poter decidere se possono partecipare studenti solo del proprio corso o del proprio dipartimento.
    Deve essere possibile creare, cancellare e modificare i poli universitari ed i corsi a loro afferenti.
    La piattaforma deve offrire una schermata di resoconto dei risultati adempiuti per ogni corso, dipartimento e polo universitario. (Chiaramente in base al fatto che le elezioni siano ).
    La visualizzazione deve essere suddivisa in corsi.
    Ad ogni votazione si dovranno scegliere le metodologie di votazione (D'Hondt, maggioritario, proporzionale, ecc), i candidati e le posizioni di questi ultimi. 
    Si deve prevedere un menù "candidati" dove poter inserire le candidature degli studenti, prevedendo tutte le cariche possibili (CCS, CDD, CSU, CNSU, CDA, CDA, ERSU) e le eventuali custom che si metteranno.
    I voti di staff dovranno essere categorizzati come tali.
    Sarà prevista anche una utility per organizzare giri aule.

    Sezione dove possiamo verificare se una lista di persone è inserita nel sistema.
}

\subsection { Stato }

\par
{
    In ideazione
}