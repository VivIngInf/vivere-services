\section { Drive }

\subsection{ Idea }

\par
{
    Utility che permette agli studenti di condividere appunti e materiale di studio.
}

\subsection { Analisi dei requisiti }

\par
{
    Tutti gli studenti potranno accedere alla piattaforma se sono stati verificati nel db.

    Gli studenti potranno visualizzare, scaricare e chiedere l'approvazione degli appunti.

    Questo vuol dire che i membri di staff dovranno controllare il materiale mandato dai ragazzi, potranno accettarlo o eliminarlo sempre con un commento costruttivo per fare capire agli studenti come contribuire alla comunità.

    I membri di staff potranno visualizzare, scaricare, aggiungere e cancellare gli appunti condivisi.

    Ogni file sarà identificato con:
    \begin{itemize}
        \item Nome del file
        \item Nome dell'autore (possibilità di lasciarlo anonimo)
        \item Data di caricamento
    \end{itemize}

    La piattaforma dovrà mantenere un log di elementi caricati, modificati e rimossi.

    La pagina iniziale dovrà permettere di selezionare qualsiasi macroarea (Ingegneria, Medicina, ecc...), dopo di che si potrà selezionare il corso, successivamente le materie (mettendo in una cartella a parte le materie non più erogate), subito dopo l'anno di erogazione ed infine la docenza.

    Ogni elemento dovrà avere dei tag che permetteranno di filtrare.
    
    I tag in questione indicheranno la tipologia: Appunti, Esercitazioni, Esercizi, Video (possibilmente anche link esterni tipo youtube), slide, documentazioni, etc \ldots

    Oltre ad un tag di tipologia ci saranno anche i tag di "ufficialità": Materiale del docente, materiale approvato dal docente, materiale non approvato dal docente.

    Inoltre ogni elemento avrà un ranking in termini di punteggio (stelle) dato dagli studenti, in questo modo gli studenti potranno sapere se in un determinato blocco di appunti manca qualcosa o ci sono degli errori, oppure se è ancora pertinente per poter preparare la materia.

    Insieme al punteggio, gli utenti potranno lasciare dei commenti e questi verranno moderati dagli amministratori, cancellandoli o approvandoli.

    Se dalla pubblicazione del commento passerà più di una settimana, il commento verrà approvato automaticamente.

    I file si potranno anche sostituire, in modo tale da aggiornare le criticità. 
    
    Per questo bisogna anche tenere traccia dei cambiamenti (stile git con i commenti dei commit).

    Gli utenti potranno anche flaggare dei commenti come offensivi, ed in quel caso potrebbero essere prese azioni disciplinari contro l'offendente.

}

\paragraph{ REGOLE }
\par
{
    Alcune regole per tutelare la piattaforma devono essere messe in atto, infatti chi trasgredirà questo regolamento potrebbe venire bannato da altri amministratori.
   
    \begin{itemize}
        \item Non sarà permesso caricare video-lezioni dei docenti e registrazioni senza consenso esplicito con email formale inoltrata ai gestori.
        \item Gli appunti non si potranno caricare come file word ma solo come PDF filigranato (vedesi utility di filigrana)
        \item Le dispense dei professori non si dovranno filigranare.
        \item Si dovrà promuovere un contesto di collaborazione ed ogni forma di hate-speech sarà punità con il ban.
    \end{itemize}
    
}

\paragraph{ EXTRA }
\par
{
    Sarebbe interessante implementare un sistema di moderazione AI dei commenti.
    Sarebbe interessante prevedere una barra di ricerca che permetta di navigare il file system più velocemente.
    Potrebbe essere carino incentivare il submitting di materiale magari dando accesso a sconti tramite kaffettino? O comunque dei premi / badge. In questo caso ogni studente potrebbe avere il proprio profilo e quindi li potremmo fare visualizzare i suoi contributi.
}

\subsection { Stato }

\par
{
    In ideazione
}